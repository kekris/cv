\documentclass[letterpaper, 10pt]{article}
%!TeX program = xelatex
\usepackage[utf8]{inputenc}
\usepackage{amsmath}
\usepackage{amssymb}
\usepackage[usenames]{color}
\usepackage{marvosym}
\usepackage{multicol}
\usepackage{parcolumns}
\usepackage[strict]{changepage}
\usepackage{hyperref}
\usepackage{xcolor}
\usepackage{academicons}
\definecolor{orcidlogocol}{HTML}{A6CE39}
\newcommand{\orcid}[1]{\href{https://orcid.org/#1}{\textcolor[HTML]{A6CE39}{\aiOrcid}}}
% \usepackage[left=0.2in, right=0.2in, bottom=1.25in, top=1.25in]{geometry}

\usepackage{setspace}
\usepackage{hyperref}

% \setlength{\tabcolsep}{0in}
\usepackage{tabularx}

\leftmargin=0.25in
\oddsidemargin=0.25in
\textwidth=6.0in
\topmargin=-0.5in
\textheight=9.25in

\raggedright

\pagenumbering{arabic}
% \thispagestyle{empty}

\def\bull{\vrule height 0.8ex width .7ex depth -.1ex }
% DEFINITIONS FOR RESUME

\newenvironment{changemargin}[2]{%
  \begin{list}{}{%
      \setlength{\topsep}{0pt}%
      \setlength{\leftmargin}{#1}%
      \setlength{\rightmargin}{#2}%
      \setlength{\listparindent}{\parindent}%
      \setlength{\itemindent}{\parindent}%
      \setlength{\parsep}{\parskip}%
    }%
  \item[]}{\end{list}
}

\newcommand{\lineover}{
  \begin{changemargin}{-0.05in}{-0.05in}
    \vspace*{-8pt}
    \hrulefill \\
    \vspace*{-2pt}
  \end{changemargin}
}

\newcommand{\header}[1]{
  \begin{changemargin}{-0.5in}{-0.5in}
    \textbf{\scshape{#1}}\\
    \lineover
  \end{changemargin}
}

\newcommand{\contact}[3]{
  \begin{changemargin}{-0.5in}{-0.5in}
    \begin{center}
      {\LARGE \textbf {#1}} % \\ \smallskip
      % {#2}\\ \smallskip
      % {#3}\smallskip
    \end{center}
  \end{changemargin}
}

\newenvironment{body} {
  \vspace*{-16pt}
  \begin{changemargin}{-0.25in}{-0.5in}
  }
  {\end{changemargin}
}

\newcommand{\school}[4]{
  \textbf{#1} \hfill \emph{#2\\}
  #3\\
  #4\\
}

% for I and II, etc.
\newcommand{\rom}[1]{\uppercase\expandafter{\romannumeral #1\relax}}

% consistent with R manual
\newcommand{\pkg}[1]{{\normalfont\fontseries{b}\selectfont #1}}
\let\proglang=\textsf
\let\code=\texttt


% END RESUME DEFINITIONS


\begin{document}
%%%%%%%%%%%%%%%%%%%%%%%%%%%%%%%%%%%%%%%%%%%%%%%%%%%%%%%%%%%%%%%%%%%%%%%%%%%%%%%%
% Name
\begin{center}
  {\huge \textbf {Junhao Ke } }\\
%  { ORCID\orcid{0000-0002-1177-1834}: 0000-0002-1177-1834}
\end{center}

% \begin{changemargin}{-0.1in}{-0.1in}
\begin{parcolumns}[]{2}
  \colchunk[1]{%
    \begin{adjustwidth}{-0.8in}{}
    \begin{itemize}
	\item[\orcid{0000-0002-1177-1834}] \href{https://orcid.org/0000-0002-1177-1834}{0000-0002-1177-1834}
	\item[\Email] \href{mailto:junhao.ke@sydney.edu.au}{junhao.ke@sydney.edu.au}
    \item[\Mobilefone] (+61)~0~451~559~391	
    \item[] Faculty of Engineering and Information Technology
    \item[] The University of Sydney	
    \item[] New South Wales 2006
    \end{itemize}
  \end{adjustwidth}
  }
  \colchunk[2]{%
    \begin{adjustwidth}{1.8in}{-2in}
    \begin{itemize}
    \end{itemize}
  \end{adjustwidth}
  }

\end{parcolumns}
% \end{changemargin}
\medskip



%%%%%%%%%%%%%%%%%%%%%%%%%%%%%%%%%%%%%%%%%%%%%%%%%%%%%%%%%%%%%%%%%%%%%%%%%%%%%%%%
% Education
\header{Education}
\begin{body}
  \vspace{14pt}
  \textbf{The University of Sydney} \hfill{NSW, Australia}\\
  \smallskip
  \emph{Doctor of Philosophy} \hfill \emph{March 2017 -- May 2021}\\
  Thesis: Direct numerical simulation of an unsteady natural convection boundary layer\\
  Advisors: Dr.\ Nicholas Williamson \& Prof.\ Steven Armfield \\
  \smallskip
  \textbf{The University of Sydney} \hfill{NSW, Australia}\\
  \smallskip
  \emph{Master of Professional Engineering} \hfill \emph{March 2015 -- December 2016}\\
  Advisors: Dr.\ Nicholas Williamson \& Prof.\ Steven Armfield \\
  \smallskip
  \textbf{East China University of Science and Techonology} \hfill{Shanghai, China}\\
  \smallskip
  \emph{Bachelor of Engineering} \hfill
  \emph{September 2010 -- July 2014} \\
\end{body}
\smallskip



\header{Research Interests}
Buoyant Driven Flows, Heat Transfer, Computational Fluid Dynamics, Statistical Computing, Turbulence, Boundary Layer Theory
\smallskip



%%%%%%%%%%%%%%%%%%%%%%%%%%%%%%%%%%%%%%%%%%%%%%%%%%%%%%%%%%%%%%%%%%%%%%%%%%%%%%%%
% Publication
\header{Publications}
\begin{body}
  \vspace{14pt}
    
  \textbf{Ke, J.}, Williamson, N., Armfield, S. W., \& Komiya, A. (2023).  The turbulence development of a vertical natural convection boundary layer. \emph{Journal of Fluid Mechanics}, xxx, A.
  
  \smallskip
  
  \textbf{Ke, J.}, Williamson, N., Armfield, S. W., Komiya, A., \& Norris, S. E. (2021).  High Grashof number turbulent natural convection on an infinite vertical wall. \emph{Journal of Fluid Mechanics}, 929, A15.
  
  \smallskip
  
  \textbf{Ke, J.}, Williamson, N., Armfield, S. W., Norris, S. E., \& Komiya, A.  (2020). Law of the wall for a temporally evolving vertical natural convection boundary layer. \emph{Journal of Fluid Mechanics}, 902, A31.
  
  \smallskip

  \textbf{Ke, J.}, , Williamson, N., Armfield, S. W., McBain, G. D., \& Norris, S. E. (2019). Stability of a temporally evolving natural convection boundary layer on an isothermal wall. \emph{Journal of Fluid Mechanics}, 877, 1163-1185.

  \smallskip

  \textbf{Ke, J.}, Williamson, N., Armfield, S. W., Norris, S. E., \& Kirkpatrick, M. (2018). Direct numerical simulation of a temporally developing natural convection boundary layer on a doubly-infinite isothermal wall, \emph{In Proceedings of IHTC-16. Begell House}. %In 16th International Heat Transfer Conference, Beijing, China, 10-15 August 2018.
  

\end{body}
\smallskip


%%%%%%%%%%%%%%%%%%%%%%%%%%%%%%
% paper in preparation (work in progress)
\header{Work in Progress}
\begin{body}
  \vspace{14pt}

  \textbf{Ke, J.}, Williamson, N., Armfield, S. W., Komiya, A., \& Norris, S. E.  Linear stability analysis of non-Boussinesq natural convection boundary layer with variable properties. (Submitted to \emph{Journal of Fluid Mechanics})
\end{body}
\smallskip

%%%%%%%%%%%%%%%%%%%%%%%%%%%%%%%%%%%%%%%%%%%%%%%%%%%%%%%%%%%%%%%%%%%%%%%%%%%%%%%%
% Talks
\header{Conferences \& Talks}
\begin{body}
  \vspace{14pt}
  
  The classical turbulent regime and Grashof number influences on the turbulence statistics of an unsteady natural convection boundary layer. In 23rd Australasian Fluid Mechanics Conference, Sydney, NSW Australia, 4-8 December 2022
  
  \smallskip
  
  {\bf Keynote: } On the classical and ultimate turbulent regimes of a natural convection boundary layer,In 12th Australasian Heat and Mass Transfer Conference, Sydney, NSW Australia, 30 June-1 July 2022
  
  \smallskip

    Turbulence statistics in a temporally evolving turbulent natural convection boundary layer. In 18th International Conference on Flow Dynamics, Sendai, Miyagi Japan, 28-29 October 2021.

  \smallskip

  Integral modelling of an unsteady natural convection boundary layer. In 22nd Australasian Fluid Mechanics Conference, Brisbane, QLD Australia, 7-10 December 2020.
  
  \smallskip
  
  Application of an integral model to an unsteady natural convection boundary layer. In 11th Australasian Natural Convection Workshop, Sydney, NSW Australia, 9-10 December 2019.
  
  \smallskip
  
  DNS of a temporally evolving vertical natural convection boundary layer. In 17th European Turbulence Conference, Torino, Italy, 3-6 September 2019.
  
  \smallskip
  
  {\bf Invited talk: } DNS study of a parallel vertical natural convection boundary layer. In Australia-Japan Fluid Dynamics Workshop, Sydney, NSW Australia, 31 January-1 February 2019.
  
  \smallskip
  
  {\bf Invited talk: } On the numerical simulation of a natural convection boundary layer on a doubly-infinite isothermal wall. In the Centre of Wind, Waves and Water, Sydney, NSW Australia, 22 June 2018.
  
  Direct numerical simulation of a temporally developing natural convection boundary layer on a doubly-infinite isothermal wall. In 16th International Heat Transfer Conference, Beijing, China, 10-15 August 2018. 
  
  \smallskip
  
  Direct numerical simulation of an unsteady natural convection boundary layer adjacent to a doubly-infinite isothermal wall. In 10th Australasian Natural Convection Workshop, Auckland, New Zealand, 30 November-1 December 2017.

\end{body}
\smallskip





%%%%%%%%%%%%%%%%%%%%%%%%%%%%%%%%%%%%%%%%%%%%%%%%%%%%%%%%%%%%%%%%%%%%%%%%%%%%%%%%
% Honors and Awards
\header{Honors \& Awards}
\begin{body}
	\vspace{14pt}
	\textbf{Postgraduate Research Support Scheme}, Faculty of Engineering and IT, USyd\hfill
	\emph{2018, 2020, 2021}\\ \smallskip
	\textbf{Charles Kolling Travelling Fund}, Faculty of Engineering and IT, USyd\hfill
	\emph{2019}\\ \smallskip
	\textbf{Best Student Paper Award}
	in 10th Australasian Natural Convection Workshop
	\hfill \emph{2017}\\ \smallskip
	\textbf{Natural Convection Supplementary Scholarship}, Faculty of Engineering and IT, USyd \hfill
	\emph{2016}\\ \smallskip
	\textbf{USyd-IS Strategic Scholarship Award}, 
	USyd \hfill \emph{2016}\\
	\smallskip
	\textbf{Dean's Excellency Award}, Faculty of Engineering and IT, USyd \hfill \emph{2015}\\
	\textbf{Merit Academic Award}, Faculty of Engineering and IT, USyd \hfill \emph{2015}\\
	\textbf{Third Prize Scholarship}, East China University of Science and Technology \hfill \emph{2014}\\
	\textbf{Fei-yang Award}, East China University of Science and Technology \hfill \emph{2014}\\
	
	
\end{body}
\smallskip
%%%%%%%%%%%%%%%%%%%%%%%%%%%%%%%%%%%%%%%%%%%%%%%%%%%%%%%%%%%%%%%%%%%%%%%%%%%%%%%%
% Teaching Positions
\header{Teaching Experience}
\begin{body}
	\vspace{14pt}
	\textbf{Teaching Assistant} \hfill \emph{March 2017 -- Present}\\
	Faculty of Engineering and IT, USyd \hfill NSW
	\vspace{-2pt}
	\begin{itemize}
		\setlength{\itemindent}{0in}
		\setlength{\itemsep}{0in}
		\item Deliver tutorial and lead discussion sessions to reinforce material covered in
		lectures. Supervise quizzes and evaluate student assignments, quizzes, exams, and other assessments.
		Course includes: Fluid Dynamics II (MECH3261), Thermal Engineering II (MECH3260), Advanced Computational Fluid Dynamics (AMME5202)
	\end{itemize}

\smallskip

	\textbf{Lecturer/Unit of Study Coordinator} \hfill \emph{February 2022 -- Present}\\
	Faculty of Engineering and IT, USyd \hfill NSW
	\vspace{-2pt}
	\begin{itemize}
		\setlength{\itemindent}{0in}
		\setlength{\itemsep}{0in}
		\item Deliver lectures and coordinate the UoS. This includes providing the teaching materials and resources, as well as administering the assessments.
		Course includes: Engineering Analysis/Biomedical Engineering 2 (AMME2000/BMET2960/BMET9960)
	\end{itemize}
\end{body}
\smallskip
%%%%%%%%%%%%%%%%%%%%%%%%%%%%%%%%%%%%%%%%%%%%%%%%%%%%%%%%%%%%%%%%%%%%%%%%%%%%%%%%
% Research Experience
\header{Research Experience}
\begin{body}
	\vspace{14pt}
	\textbf{Postdoctoral Research Associate}\hfill
	\emph{March 2021 -- Present} \\
	School of Aerospace, Mechanical and Mechatronic Engineering, The University of Sydney   \hfill NSW, Australia
	\vspace{-2pt}
	\begin{itemize}
		\setlength{\itemindent}{0in}
		\setlength{\itemsep}{0in}
		\item Turbulent flows; Direct numerical simulation; Buoyancy induced flows; Boundary layers 
%		\item Statistical analysis for the data obtained
	\end{itemize}

	\vspace{14pt}
	\textbf{Visiting Researcher}\hfill
	\emph{September 2019 -- October 2019} \\
	Advanced Fluid Information Research Center, Institute of Fluid Science, Tohoku University   \hfill Sendai, Japan
	\vspace{-2pt}
	\begin{itemize}
		\setlength{\itemindent}{0in}
		\setlength{\itemsep}{0in}
		\item International cooperation on the natural convection/ventilation project 
%		\item 
	\end{itemize}

	
\end{body}
\smallskip
%%%%%%%%%%%%%%%%%%%%%%%%%%%%%%%%%%%%%%%%%%%%%%%%%%%%%%%%%%%%%%%%%%%%%%%%%%%%%%%%
% Work Experience
\header{Industry Experience}
\begin{body}
	\vspace{14pt}
	
	\textbf{Project Engineer}\hfill
	\emph{November 2015 -- February 2016} \\
	Department of Research \& Development, Inalfa Co., Ltd.  \hfill Shanghai, China
	\vspace{-2pt}
	\begin{itemize}
		\setlength{\itemindent}{0in}
		\setlength{\itemsep}{0in}
		\item Experiment design \& validation
		\item Statistical analysis for experimental data
		\item Algorithm development for acoustic analysis programs
	\end{itemize}
	
	
	\textbf{Assistant Manager}\hfill
	\emph{June 2014 -- December 2014} \\
	Department of Construction \& Excavation Machinery, Yanmar Engines Co.,  \hfill Shanghai, China
	\vspace{-2pt}
	\begin{itemize}
		\setlength{\itemindent}{0in}
		\setlength{\itemsep}{0in}
		\item Statistical analysis for recurrent event data
		\item Inventory control
	\end{itemize}
	

\end{body}
\smallskip

%%%%%%%%%%%%%%%%%%%%%%%%%%%%%%%%%%%%%%%%%%%%%%%%%%%%%%%%%%%%%%%%%%%%%%%%%%%%%%%%
% Service
\header{Service}
\begin{body}
	\vspace{14pt}
	
	\textbf{Volunteer} of China Open Day (USyd) \hfill
	\emph{2015}
	\vspace{-2pt}
	\begin{itemize}
		\setlength{\itemindent}{0in}
		\setlength{\itemsep}{0in}
		\item Providing assistance on behalf of the faculty of Engineering and IT with the USyd global student recruitment team. 
	\end{itemize}
	

\end{body}
\smallskip
%%%%%%%%%%%%%%%%%%%%%%%%%%%%%%%%%%%%%%%%%%%%%%%%%%%%%%%%%%%%%%%%%%%%%%%%%%%%%%%%
% Language
 \header{Language}

 \begin{body}
   \vspace{14pt}
   \textbf{English} (fluent), \textbf{Japanese} (fluent), \textbf{Mandarin} (native) and \textbf{Shanghai Dialect} (native) \\
 \end{body}




\end{document}
